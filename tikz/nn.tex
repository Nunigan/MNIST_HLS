\documentclass[border=10pt]{standalone}
\usepackage[left=25mm,right=25mm,top=25mm,bottom=25mm]{geometry}
\usepackage[utf8]{inputenc}
\usepackage[T1]{fontenc}
\usepackage{times}
\usepackage{geometry}
\usepackage{amsmath}
\usepackage{amssymb}
\usepackage{mathrsfs}
\usepackage{amsfonts}
\usepackage{amsthm}
\usepackage{lipsum}
\usepackage{amscd}
\usepackage{graphicx}
\usepackage{fancyhdr}
\usepackage{textcomp}
\usepackage{txfonts}
\usepackage[all]{xy}
\usepackage{paralist}
\usepackage[colorlinks=true]{hyperref}
\usepackage{array}
\usepackage{tikz}
\usepackage{slashed}
\usepackage{pdfpages}
\usepackage{cite}
\usepackage{url}
\usepackage{amsmath,amsfonts,amssymb}
\usepackage{tikz}
\usetikzlibrary{automata,positioning}
\usepackage{listings}
\usepackage{multirow}
\usepackage{color}
\usetikzlibrary{arrows,decorations.pathreplacing}
\begin{document}


% Input layer neurons'number
\newcommand{\inputnum}{7}

% Hidden layer neurons'number
\newcommand{\hiddennumA}{3}
\newcommand{\hiddennumB}{5}

% Output layer neurons'number
\newcommand{\outputnum}{4}

\begin{tikzpicture}

% \node[] at (0,0) {784};
% \node[] at (2.5,0) {128};
% \node[] at (5,0) {256};
% \node[] at (7.5,0) {10};

\node [] at (0, -8) {layer 1 (Input)};
\node [] at (5, -8) {layer $l$};
\node [] at (10, -8) {layer $L$ (Output)};
\draw[->] (6,-1) node[anchor=south] {Outputs $a_1(l)$} -- (5.4,-1.8);
\draw[->] (6,-6.9) node[anchor=north] {Outputs $a_{n_l}(l)$} -- (5.4,-6.3);



% Input Layer
\foreach \i in {1,...,\inputnum}
{
	\node[circle,
		minimum size = 6mm,
		draw] (Input-\i) at (0,-\i) {};
}

% Hidden Layer1
\foreach \i in {1,...,\hiddennumA}
{
	\node[circle,
		minimum size = 6mm,
		draw,
		yshift=(\hiddennumA-\inputnum)*5 mm
	] (Hidden1-\i) at (2.5,-\i) {};
}
% \node [] at (2.5, -7) {layer 1};


% Hidden Layer2
\foreach \i in {1,...,\hiddennumB}
{
	\node[circle,
		minimum size = 6mm,
		draw,
		yshift=(\hiddennumB-\inputnum)*5 mm
	] (Hidden2-\i) at (5,-\i) {};
}



% Hidden Layer3
\foreach \i in {1,...,\hiddennumB}
{
	\node[circle,
		minimum size = 6mm,
		draw,
		yshift=(\hiddennumB-\inputnum)*5 mm
	] (Hidden3-\i) at (7.5,-\i) {};
}


% \node [] at (7.5, -7) {layer $l$};

% Output Layer
\foreach \i in {1,...,\outputnum}
{
	\node[circle,
		minimum size = 6mm,
		draw,
		yshift=(\outputnum-\inputnum)*5 mm
	] (Output-\i) at (10,-\i) {};
}

% Connect neurons In-Hidden
\foreach \i in {1,...,\inputnum}
{
	\foreach \j in {1,...,\hiddennumA}
	{
		\draw[->] (Input-\i) -- (Hidden1-\j);
	}
}

% Connect neurons In-Hidden
\foreach \i in {1,...,\hiddennumA}
{
	\foreach \j in {1,...,\hiddennumB}
	{
		\draw[->] (Hidden1-\i) -- (Hidden2-\j);
	}
}

% Connect neurons Hidden-Out
\foreach \i in {1,...,\hiddennumB}
{
	\foreach \j in {1,...,\hiddennumB}
	{
		\draw[->] (Hidden2-\i) -- (Hidden3-\j);
	}
}

\foreach \i in {1,...,\hiddennumB}
{
	\foreach \j in {1,...,\outputnum}
	{
		\draw[->] (Hidden3-\i) -- (Output-\j);
	}
}

% Inputs
\foreach \i in {1,...,6}
{
	\draw[<-] (Input-\i) -- ++(-1,0)
		node[left]{$x_{\i}$};
}

\draw[<-] (Input-7) -- ++(-1,0)
	node[left]{$x_{n_1}$};

% Outputs
\foreach \i in {1,...,3}
{
	\draw[->] (Output-\i) -- ++(1,0)
		node[right]{$y_{\i}$};
}
\draw[->] (Output-4) -- ++(1,0)
	node[right]{$y_{n_L}$};
	
	\draw[thick,decorate,decoration={brace,amplitude=3pt,mirror}] (2,-8.3) -- (8,-8.3);
	\node[] at (5,-8.7) {Hidden Layers}; 

	
\end{tikzpicture}


\end{document}
